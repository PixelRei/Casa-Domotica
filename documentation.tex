\documentclass[a4paper,12pt]{article}
\usepackage[utf8]{inputenc}
\usepackage[T1]{fontenc}
\usepackage[italian]{babel}
\usepackage{graphicx}
\usepackage{geometry}
\usepackage{fancyhdr}
\usepackage{booktabs}
\usepackage{multicol}
\usepackage[colorlinks=true,
            linkcolor=black,
            citecolor=black,
            urlcolor=blue]{hyperref}  % Configurazione hyperref per link ipertestuali

% Impostazioni pagina
\geometry{a4paper, margin=2.5cm}
\pagestyle{fancy}
\fancyhf{}
\renewcommand{\headrulewidth}{0pt}

% Pagina del titolo
\title{Casa Domotica}
\author{Team di Sviluppo}
\date{}

\begin{document}

% Pagina di Copertina
\begin{titlepage}
    \begin{minipage}[t]{0.4\textwidth}
    \raggedright
    {\large\itshape Team di Sviluppo:\par}
    \vspace{0.2cm}
    
    Cociug Raul Andrei - Sviluppatore \\
    Madiotto Gabriel - Sviluppatore
    \end{minipage}
    \hfill
    \begin{minipage}[t]{0.4\textwidth}
    \raggedleft
    {\Large Azienda: DomotiX\par}
    \end{minipage}
    
    \centering
    \vspace*{5cm}
    
    {\huge\bfseries Casa Domotica\par}
    \vspace{1.5cm}
    
    \vfill
    
    % Footer con tabella versioni
    \begin{table}[h]
    \centering
    \begin{tabular}{@{}lllc@{}}
    \toprule
    Versione & Data & Autore & Docenti \\
    \midrule
    1.0 & ? & DomotiX(?) & Tollot, Rossi \\
    \bottomrule
    \end{tabular}
    \end{table}
    
    \thispagestyle{empty}
\end{titlepage}

% Resto del documento rimane invariato
\tableofcontents
\newpage

\fancyfoot[C]{\thepage}

% Il resto del documento rimane esattamente come nel documento originale (sezioni precedenti)
\section{Introduzione}
\subsection{Descrizione Generale del Progetto}
L'obiettivo del progetto è quello di sviluppare un sistema che permetta il controllo e il monitoraggio di vari dispositivi domestici attraverso un'interfaccia utente interattiva.
Il progetto si compone di due principali elementi:\begin{itemize}
    \item \textbf{La rete della casa domotica}, realizzata mediante il software Cisco Packet Tracer, per simulare la comunicazione tra i vari dispositivi connessi.
    \item \textbf{L'interfaccia utente}, sviluppata con HTML, CSS e JavaScript, per consentire l'interazione con i dispositivi attraverso un mockup funzionale.\end{itemize}

\subsection{Informazioni Progetto}
\begin{itemize}
    \item \textbf{Data Inizio Progetto:} 11 Marzo 2025
    \item \textbf{Data Fine Prevista:} 14 Maggio 2025
    \item \textbf{Durata Stimata:} 65 giorni/ 9 settimane
    \item \textbf{Partecipanti:}
    \begin{itemize}
        \item Cociug Raul
        \item Madiotto Gabriel
    \end{itemize}
\end{itemize}

% Resto del documento come nell'originale (sezioni successive)
\section{Situazione Attuale}
\subsection{Stato Corrente}
È stato sviluppato il mockup con HTML, CSS, JavaScript per mostrare l'anteprima dell'interfaccia utente che dovrà essere utilizzata dagli utenti per poter interagire con la casa domotica e i dispositivi connessi ad essa.
È stato realizzato un pannello di controllo, per la casa e tutte le stanze, con il quale è possibile svolgere numerose funzioni come:\begin{itemize}
\item Accendere/spegnere le luci
\item Utilizzo delle telecamere
\item Controllo della temperatura all'esterno
\item Controllo della velocità del vento
\item Verifica dello stato delle porte (aperte/chiuse)
\end{itemize}
È stato realizzato anche il questionario da dover porre ai clienti per l'analisi dei requisiti e delle funzionalità da implementare per personalizzare il prodotto come richiesto dal cliente.

\subsection{Materiale Esistente}
\begin{itemize}
\item \textbf{Mockup/Interfaccia utente}
\item \textbf{Questionario}
\item \textbf{Documentazione del progetto}
\end{itemize}

\section{Requisiti Hardware e Software}
\subsection{Requisiti del Cliente}
\begin{itemize}
    \item \textbf{Postazione Computer:}
    \begin{itemize}
        \item Processore: 
        \item RAM: 
        \item Spazio Disco:
    \end{itemize}
    
    \item \textbf{Connettività Internet:}
    \begin{itemize}
        \item Velocità Minima:
        \item Tipo Connessione:
    \end{itemize}
\end{itemize}

\section{Requisiti Funzionali}
\subsection{Richieste Cliente HMI}
Descrizione dettagliata delle richieste dell'interfaccia uomo-macchina.

\subsection{Specifiche Funzionali}
Elencare dettagliatamente cosa si vuole ottenere con il progetto.

\subsection{Progettazione Interfaccia Grafica (GUI)}
\begin{itemize}
    \item Layout
    \item Elementi Grafici
    \item Esperienza Utente
\end{itemize}

\subsection{Progettazione Rete}
\begin{itemize}
    \item Architettura di Rete
    \item Protocolli
    \item Sicurezza
\end{itemize}

\end{document}
