\documentclass[a4paper,12pt]{article}
\usepackage[utf8]{inputenc}
\usepackage[T1]{fontenc}
\usepackage[italian]{babel}
\usepackage{graphicx}
\usepackage{geometry}
\usepackage{fancyhdr}
\usepackage{booktabs}
\usepackage[colorlinks=true,
            linkcolor=black,
            citecolor=black,
            urlcolor=blue]{hyperref}  % Configurazione hyperref per link ipertestuali

% Impostazioni pagina
\geometry{a4paper, margin=2.5cm}
\pagestyle{fancy}
\fancyhf{}
\renewcommand{\headrulewidth}{0pt}

% Pagina del titolo
\title{Casa Domotica}
\author{Team di Sviluppo}
\date{}

\begin{document}

% Pagina di Copertina
\begin{titlepage}
    \centering
    \vspace*{2cm}
    
    {\huge\bfseries Casa Domotica\par}
    \vspace{1.5cm}
    
    {\Large Azienda: DomotiX\par}
    \vspace{2cm}
    
    {\large\itshape Team di Sviluppo:\par}
    \vspace{0.5cm}
    
    % Elenco dei partecipanti
    \begin{itemize}
        \item Cociug Raul Andrei - Sviluppatore
        \item Madiotto Gabriel - Sviluppatore
    \end{itemize}
    
    \vfill
    
    % Footer con tabella versioni
    \begin{table}[h]
    \centering
    \begin{tabular}{@{}lllc@{}}
    \toprule
    Versione & Data & Autore & Docenti \\
    \midrule
    1.0 & ? & DomotiX(?) & Tollot, Rossi \\
    \bottomrule
    \end{tabular}
    \end{table}
    
    \thispagestyle{empty}
\end{titlepage}

% Indice con link
\tableofcontents
\newpage

% Intestazione per il footer
\fancyfoot[C]{\thepage}

\section{Introduzione}
\subsection{Descrizione Generale del Progetto}
Breve descrizione del progetto software/prodotto che si sta sviluppando. !!!!

\subsection{Informazioni Progetto}
\begin{itemize}
    \item \textbf{Data Inizio Progetto:} 11 Marzo 2025
    \item \textbf{Data Fine Prevista:} 14 Maggio 2025
    \item \textbf{Durata Stimata:} 65 giorni/ 9 settimane
    \item \textbf{Partecipanti:}
    \begin{itemize}
        \item Cociug Raul
        \item Madiotto Gabriel
    \end{itemize}
\end{itemize}

\section{Situazione Attuale}
\subsection{Stato Corrente}
Descrizione dettagliata di ciò che è stato fatto finora nel progetto.

\subsection{Materiale Esistente}
Elencare eventuali risorse, mockup, prototipi già sviluppati.

\section{Requisiti Hardware e Software}
\subsection{Requisiti del Cliente}
\begin{itemize}
    \item \textbf{Postazione Computer:}
    \begin{itemize}
        \item Processore: 
        \item RAM: 
        \item Spazio Disco:
    \end{itemize}
    
    \item \textbf{Connettività Internet:}
    \begin{itemize}
        \item Velocità Minima:
        \item Tipo Connessione:
    \end{itemize}
\end{itemize}

\section{Requisiti Funzionali}
\subsection{Richieste Cliente HMI}
Descrizione dettagliata delle richieste dell'interfaccia uomo-macchina.

\subsection{Specifiche Funzionali}
Elencare dettagliatamente cosa si vuole ottenere con il progetto.

\subsection{Progettazione Interfaccia Grafica (GUI)}
\begin{itemize}
    \item Layout
    \item Elementi Grafici
    \item Esperienza Utente
\end{itemize}

\subsection{Progettazione Rete}
\begin{itemize}
    \item Architettura di Rete
    \item Protocolli
    \item Sicurezza
\end{itemize}

\end{document}
